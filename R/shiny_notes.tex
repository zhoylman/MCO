\documentclass[]{article}
\usepackage{lmodern}
\usepackage{amssymb,amsmath}
\usepackage{ifxetex,ifluatex}
\usepackage{fixltx2e} % provides \textsubscript
\ifnum 0\ifxetex 1\fi\ifluatex 1\fi=0 % if pdftex
  \usepackage[T1]{fontenc}
  \usepackage[utf8]{inputenc}
\else % if luatex or xelatex
  \ifxetex
    \usepackage{mathspec}
  \else
    \usepackage{fontspec}
  \fi
  \defaultfontfeatures{Ligatures=TeX,Scale=MatchLowercase}
\fi
% use upquote if available, for straight quotes in verbatim environments
\IfFileExists{upquote.sty}{\usepackage{upquote}}{}
% use microtype if available
\IfFileExists{microtype.sty}{%
\usepackage{microtype}
\UseMicrotypeSet[protrusion]{basicmath} % disable protrusion for tt fonts
}{}
\usepackage[margin=1in]{geometry}
\usepackage{hyperref}
\hypersetup{unicode=true,
            pdftitle={A Tutorial on Shiny and Leaflet},
            pdfauthor={Zachary Hoylman},
            pdfborder={0 0 0},
            breaklinks=true}
\urlstyle{same}  % don't use monospace font for urls
\usepackage{color}
\usepackage{fancyvrb}
\newcommand{\VerbBar}{|}
\newcommand{\VERB}{\Verb[commandchars=\\\{\}]}
\DefineVerbatimEnvironment{Highlighting}{Verbatim}{commandchars=\\\{\}}
% Add ',fontsize=\small' for more characters per line
\usepackage{framed}
\definecolor{shadecolor}{RGB}{248,248,248}
\newenvironment{Shaded}{\begin{snugshade}}{\end{snugshade}}
\newcommand{\AlertTok}[1]{\textcolor[rgb]{0.94,0.16,0.16}{#1}}
\newcommand{\AnnotationTok}[1]{\textcolor[rgb]{0.56,0.35,0.01}{\textbf{\textit{#1}}}}
\newcommand{\AttributeTok}[1]{\textcolor[rgb]{0.77,0.63,0.00}{#1}}
\newcommand{\BaseNTok}[1]{\textcolor[rgb]{0.00,0.00,0.81}{#1}}
\newcommand{\BuiltInTok}[1]{#1}
\newcommand{\CharTok}[1]{\textcolor[rgb]{0.31,0.60,0.02}{#1}}
\newcommand{\CommentTok}[1]{\textcolor[rgb]{0.56,0.35,0.01}{\textit{#1}}}
\newcommand{\CommentVarTok}[1]{\textcolor[rgb]{0.56,0.35,0.01}{\textbf{\textit{#1}}}}
\newcommand{\ConstantTok}[1]{\textcolor[rgb]{0.00,0.00,0.00}{#1}}
\newcommand{\ControlFlowTok}[1]{\textcolor[rgb]{0.13,0.29,0.53}{\textbf{#1}}}
\newcommand{\DataTypeTok}[1]{\textcolor[rgb]{0.13,0.29,0.53}{#1}}
\newcommand{\DecValTok}[1]{\textcolor[rgb]{0.00,0.00,0.81}{#1}}
\newcommand{\DocumentationTok}[1]{\textcolor[rgb]{0.56,0.35,0.01}{\textbf{\textit{#1}}}}
\newcommand{\ErrorTok}[1]{\textcolor[rgb]{0.64,0.00,0.00}{\textbf{#1}}}
\newcommand{\ExtensionTok}[1]{#1}
\newcommand{\FloatTok}[1]{\textcolor[rgb]{0.00,0.00,0.81}{#1}}
\newcommand{\FunctionTok}[1]{\textcolor[rgb]{0.00,0.00,0.00}{#1}}
\newcommand{\ImportTok}[1]{#1}
\newcommand{\InformationTok}[1]{\textcolor[rgb]{0.56,0.35,0.01}{\textbf{\textit{#1}}}}
\newcommand{\KeywordTok}[1]{\textcolor[rgb]{0.13,0.29,0.53}{\textbf{#1}}}
\newcommand{\NormalTok}[1]{#1}
\newcommand{\OperatorTok}[1]{\textcolor[rgb]{0.81,0.36,0.00}{\textbf{#1}}}
\newcommand{\OtherTok}[1]{\textcolor[rgb]{0.56,0.35,0.01}{#1}}
\newcommand{\PreprocessorTok}[1]{\textcolor[rgb]{0.56,0.35,0.01}{\textit{#1}}}
\newcommand{\RegionMarkerTok}[1]{#1}
\newcommand{\SpecialCharTok}[1]{\textcolor[rgb]{0.00,0.00,0.00}{#1}}
\newcommand{\SpecialStringTok}[1]{\textcolor[rgb]{0.31,0.60,0.02}{#1}}
\newcommand{\StringTok}[1]{\textcolor[rgb]{0.31,0.60,0.02}{#1}}
\newcommand{\VariableTok}[1]{\textcolor[rgb]{0.00,0.00,0.00}{#1}}
\newcommand{\VerbatimStringTok}[1]{\textcolor[rgb]{0.31,0.60,0.02}{#1}}
\newcommand{\WarningTok}[1]{\textcolor[rgb]{0.56,0.35,0.01}{\textbf{\textit{#1}}}}
\usepackage{graphicx,grffile}
\makeatletter
\def\maxwidth{\ifdim\Gin@nat@width>\linewidth\linewidth\else\Gin@nat@width\fi}
\def\maxheight{\ifdim\Gin@nat@height>\textheight\textheight\else\Gin@nat@height\fi}
\makeatother
% Scale images if necessary, so that they will not overflow the page
% margins by default, and it is still possible to overwrite the defaults
% using explicit options in \includegraphics[width, height, ...]{}
\setkeys{Gin}{width=\maxwidth,height=\maxheight,keepaspectratio}
\IfFileExists{parskip.sty}{%
\usepackage{parskip}
}{% else
\setlength{\parindent}{0pt}
\setlength{\parskip}{6pt plus 2pt minus 1pt}
}
\setlength{\emergencystretch}{3em}  % prevent overfull lines
\providecommand{\tightlist}{%
  \setlength{\itemsep}{0pt}\setlength{\parskip}{0pt}}
\setcounter{secnumdepth}{0}
% Redefines (sub)paragraphs to behave more like sections
\ifx\paragraph\undefined\else
\let\oldparagraph\paragraph
\renewcommand{\paragraph}[1]{\oldparagraph{#1}\mbox{}}
\fi
\ifx\subparagraph\undefined\else
\let\oldsubparagraph\subparagraph
\renewcommand{\subparagraph}[1]{\oldsubparagraph{#1}\mbox{}}
\fi

%%% Use protect on footnotes to avoid problems with footnotes in titles
\let\rmarkdownfootnote\footnote%
\def\footnote{\protect\rmarkdownfootnote}

%%% Change title format to be more compact
\usepackage{titling}

% Create subtitle command for use in maketitle
\providecommand{\subtitle}[1]{
  \posttitle{
    \begin{center}\large#1\end{center}
    }
}

\setlength{\droptitle}{-2em}

  \title{\textbf{A Tutorial on Shiny and Leaflet}}
    \pretitle{\vspace{\droptitle}\centering\huge}
  \posttitle{\par}
    \author{Zachary Hoylman}
    \preauthor{\centering\large\emph}
  \postauthor{\par}
      \predate{\centering\large\emph}
  \postdate{\par}
    \date{11/14/2019}


\begin{document}
\maketitle

\begin{center}
Montana Climate Office 
\end{center}
\begin{center}
zachary.hoylman@mso.umt.edu
\end{center}

\hypertarget{set-up-and-download-required-packages}{%
\section{Set up and download required
packages}\label{set-up-and-download-required-packages}}

Before we begin, let's install all the packages we will use throughout
this lecture.

\begin{Shaded}
\begin{Highlighting}[]
\NormalTok{libraries =}\StringTok{ }\KeywordTok{c}\NormalTok{(}\StringTok{"knitr"}\NormalTok{, }\StringTok{"shiny"}\NormalTok{, }\StringTok{"leaflet"}\NormalTok{,}\StringTok{"ncdf4"}\NormalTok{, }\StringTok{"lubridate"}\NormalTok{, }\StringTok{"dplyr"}\NormalTok{,}
                              \StringTok{"zoo"}\NormalTok{, }\StringTok{"ggplot2"}\NormalTok{, }\StringTok{"scales"}\NormalTok{, }\StringTok{"leaflet.extras"}\NormalTok{)}

\KeywordTok{install.packages}\NormalTok{(libraries)}

\KeywordTok{lapply}\NormalTok{(libraries, library, }\DataTypeTok{character.only =} \OtherTok{TRUE}\NormalTok{)}
\end{Highlighting}
\end{Shaded}

\hypertarget{shiny-introduction-and-example}{%
\section{Shiny: Introduction and
example}\label{shiny-introduction-and-example}}

Shiny is an R package that makes it easy to build interactive web
applications (apps) straight from R. This lesson will get you familiar
with Shiny apps right away.

First lets run a shiny app example that comes with shiny to understand
what apps are (in a very very very simple form).

\begin{Shaded}
\begin{Highlighting}[]
\KeywordTok{library}\NormalTok{(shiny)}
\KeywordTok{runExample}\NormalTok{(}\StringTok{"01_hello"}\NormalTok{)}
\end{Highlighting}
\end{Shaded}

\hypertarget{shiny-layout}{%
\section{Shiny layout}\label{shiny-layout}}

There are two main components to a shiny app, a User Interface (UI) and
a server. The UI is used to pass arguments to the server script which
conducts the operations. These two parts of the app can either be
contained in two separate scripts (ui.R and server.R) or can be combined
in a single script that has both components.

\begin{verbatim}
shinyApp(ui = defines the user interface,
         server = function(input,output) {
         define the operations being conducted and does things like plotting}
         )
\end{verbatim}

\hypertarget{simple-script-example}{%
\subsection{Simple script example}\label{simple-script-example}}

This script recreates the app shown above to familiarize ourselves with
how shiny apps operate. We are going to be using a base R data set for
this example, ``faithful'', which is a data set of Old Faithful geyser
eruptions.

Lets take a quick look at the data

\begin{Shaded}
\begin{Highlighting}[]
\KeywordTok{head}\NormalTok{(faithful)}
\end{Highlighting}
\end{Shaded}

\begin{verbatim}
##   eruptions waiting
## 1     3.600      79
## 2     1.800      54
## 3     3.333      74
## 4     2.283      62
## 5     4.533      85
## 6     2.883      55
\end{verbatim}

``eruptions'' represent the length of time of a single eruption, and
``waiting'' is the time in between eruptions. We are going to be using
the second column in this data set.

Now lets look at the app.

\begin{Shaded}
\begin{Highlighting}[]
\KeywordTok{shinyApp}\NormalTok{(}
  \CommentTok{# First lets build the user interface (UI).}
  \DataTypeTok{ui =} \KeywordTok{fluidPage}\NormalTok{(}
  \CommentTok{# App title ----}
  \KeywordTok{titlePanel}\NormalTok{(}\StringTok{"Old Faithful Eruptions"}\NormalTok{),}
  \CommentTok{# There are different lay out options we can use in shiny,}
  \CommentTok{# here we will be using the "sidebarLayout" option.}
  \KeywordTok{sidebarLayout}\NormalTok{(}
    \CommentTok{# Now we add sidebar panel for inputs ----}
    \KeywordTok{sidebarPanel}\NormalTok{(}
      \CommentTok{# Input: Slider for the number of bins ----}
      \KeywordTok{sliderInput}\NormalTok{(}\DataTypeTok{inputId =} \StringTok{"bins"}\NormalTok{,}
                  \DataTypeTok{label =} \StringTok{"Number of bins:"}\NormalTok{,}
                  \DataTypeTok{min =} \DecValTok{1}\NormalTok{,}
                  \DataTypeTok{max =} \DecValTok{50}\NormalTok{,}
                  \DataTypeTok{value =} \DecValTok{30}\NormalTok{)}
\NormalTok{    ),}
    \CommentTok{# Main panel for displaying outputs ----}
    \KeywordTok{mainPanel}\NormalTok{(}
      \CommentTok{# Output: Histogram named distPlot. }
      \CommentTok{# distPlot is defined below. }
      \KeywordTok{plotOutput}\NormalTok{(}\DataTypeTok{outputId =} \StringTok{"distPlot"}\NormalTok{)}
\NormalTok{    )}
\NormalTok{  )}
\NormalTok{),}
\NormalTok{server <-}\StringTok{ }\ControlFlowTok{function}\NormalTok{(input, output) \{}
  \CommentTok{# Histogram of the Old Faithful Geyser Data}
  \CommentTok{# with requested number of bins}
  \CommentTok{# This expression that generates a histogram is wrapped in a call}
  \CommentTok{# to renderPlot to indicate that:}
  \CommentTok{#}
  \CommentTok{# 1. It is "reactive" and therefore should be automatically}
  \CommentTok{#    re-executed when inputs (input$bins) change}
  \CommentTok{# 2. Its output type is a plot}
\NormalTok{  output}\OperatorTok{$}\NormalTok{distPlot <-}\StringTok{ }\KeywordTok{renderPlot}\NormalTok{(\{}
    \CommentTok{# Fist we define what data is being used in the plot. }
    \CommentTok{# For a histogram we want our variable of interest on the x}
    \CommentTok{# The user doesnt have an option to change this.}
\NormalTok{    x    <-}\StringTok{ }\NormalTok{faithful}\OperatorTok{$}\NormalTok{waiting}
    \CommentTok{# This is the part that is modifiied by the user (input$bins). }
    \CommentTok{# Here the user is overwriting the "bins" variable }
\NormalTok{    bins <-}\StringTok{ }\KeywordTok{seq}\NormalTok{(}\KeywordTok{min}\NormalTok{(x), }\KeywordTok{max}\NormalTok{(x), }\DataTypeTok{length.out =}\NormalTok{ input}\OperatorTok{$}\NormalTok{bins }\OperatorTok{+}\StringTok{ }\DecValTok{1}\NormalTok{)}
    \CommentTok{#here is the plot}
    \KeywordTok{hist}\NormalTok{(x, }\DataTypeTok{breaks =}\NormalTok{ bins, }\DataTypeTok{col =} \StringTok{"#75AADB"}\NormalTok{, }\DataTypeTok{border =} \StringTok{"white"}\NormalTok{,}
         \DataTypeTok{xlab =} \StringTok{"Time in between eruptions (Minutes)"}\NormalTok{,}
         \DataTypeTok{main =} \StringTok{"Histogram of waiting times"}\NormalTok{)}
\NormalTok{    \})}
\NormalTok{\})}
\end{Highlighting}
\end{Shaded}

\hypertarget{the-same-script-without-comments.}{%
\subsection{The same script without
comments.}\label{the-same-script-without-comments.}}

Try changing this code to instead show a histogram of the length of en
eruption rather than waiting times in between and allow the user to
choose bins from a more constricted range, say 10 - 20 with a starting
value of 15.

\begin{Shaded}
\begin{Highlighting}[]
\KeywordTok{shinyApp}\NormalTok{(}
  \DataTypeTok{ui =} \KeywordTok{fluidPage}\NormalTok{(}
  \KeywordTok{titlePanel}\NormalTok{(}\StringTok{"Old Faithful Eruptions"}\NormalTok{),}
  \KeywordTok{sidebarLayout}\NormalTok{(}
    \KeywordTok{sidebarPanel}\NormalTok{(}
      \KeywordTok{sliderInput}\NormalTok{(}\DataTypeTok{inputId =} \StringTok{"bins"}\NormalTok{,}
                  \DataTypeTok{label =} \StringTok{"Number of bins:"}\NormalTok{,}
                  \DataTypeTok{min =} \DecValTok{5}\NormalTok{,}
                  \DataTypeTok{max =} \DecValTok{10}\NormalTok{,}
                  \DataTypeTok{value =} \DecValTok{30}\NormalTok{)}
\NormalTok{    ),}
    \KeywordTok{mainPanel}\NormalTok{(}
      \KeywordTok{plotOutput}\NormalTok{(}\DataTypeTok{outputId =} \StringTok{"distPlot"}\NormalTok{)}
\NormalTok{    )}
\NormalTok{  )}
\NormalTok{),}
\NormalTok{server <-}\StringTok{ }\ControlFlowTok{function}\NormalTok{(input, output) \{}
\NormalTok{  output}\OperatorTok{$}\NormalTok{distPlot <-}\StringTok{ }\KeywordTok{renderPlot}\NormalTok{(\{}
\NormalTok{    x    <-}\StringTok{ }\NormalTok{faithful}\OperatorTok{$}\NormalTok{eruptions}
\NormalTok{    bins <-}\StringTok{ }\KeywordTok{seq}\NormalTok{(}\KeywordTok{min}\NormalTok{(x), }\KeywordTok{max}\NormalTok{(x), }\DataTypeTok{length.out =}\NormalTok{ input}\OperatorTok{$}\NormalTok{bins }\OperatorTok{+}\StringTok{ }\DecValTok{1}\NormalTok{)}
    \KeywordTok{hist}\NormalTok{(x, }\DataTypeTok{breaks =}\NormalTok{ bins, }\DataTypeTok{col =} \StringTok{"#75AADB"}\NormalTok{, }\DataTypeTok{border =} \StringTok{"white"}\NormalTok{,}
         \DataTypeTok{xlab =} \StringTok{"Time in between eruptions (Minutes)"}\NormalTok{,}
         \DataTypeTok{main =} \StringTok{"Histogram of waiting times"}\NormalTok{)}
\NormalTok{    \})}
\NormalTok{\})}
\end{Highlighting}
\end{Shaded}

Now you can see that there is a cyclic nature to these apps. The UI
passes an input to the server, the server does computation and passes an
output to the UI. UI -\textgreater{} input -\textgreater{} server
-\textgreater{} output -\textgreater{} UI

\hypertarget{multivariate-example}{%
\section{Multivariate example}\label{multivariate-example}}

Great! Now we have a better idea of how these apps work. Lets now do
something a bit more realistic. Say you have a lot of data that you want
to visualize in different ways. Lets make an app that allows the user to
choose x and y variables from a list of variables, plots them and does a
bit of stats. We are going to use the dataset ``iris'' for this example.
This dataset has information about the properties of flowers.

\begin{Shaded}
\begin{Highlighting}[]
\KeywordTok{head}\NormalTok{(iris)}
\end{Highlighting}
\end{Shaded}

\begin{verbatim}
##   Sepal.Length Sepal.Width Petal.Length Petal.Width Species
## 1          5.1         3.5          1.4         0.2  setosa
## 2          4.9         3.0          1.4         0.2  setosa
## 3          4.7         3.2          1.3         0.2  setosa
## 4          4.6         3.1          1.5         0.2  setosa
## 5          5.0         3.6          1.4         0.2  setosa
## 6          5.4         3.9          1.7         0.4  setosa
\end{verbatim}

\hypertarget{the-app}{%
\subsection{The app}\label{the-app}}

This app is going to have 3 user inputs and output a plot. We are going
to let the user choose an X dataset to plot against a Y dataset and then
calculate a linear model to display the relationship between them.
Finally we will allow the user to choose a polynomial degree to modify
the linear model's shape.

\begin{Shaded}
\begin{Highlighting}[]
\KeywordTok{shinyApp}\NormalTok{(}\DataTypeTok{ui =} \KeywordTok{fluidPage}\NormalTok{(}
  \CommentTok{# First we will build the UI in the same fashion as before}
  \KeywordTok{titlePanel}\NormalTok{(}\StringTok{"Flowers!"}\NormalTok{),}
  \CommentTok{# we are going to allow the user to choose a polynomial degree for the model}
  \KeywordTok{numericInput}\NormalTok{(}\DataTypeTok{inputId =} \StringTok{"poly_degree"}\NormalTok{,}
                  \DataTypeTok{label =} \StringTok{"Choose Polynomial Degree"}\NormalTok{,}
                  \DataTypeTok{value =} \DecValTok{1}\NormalTok{, }\DataTypeTok{min =} \DecValTok{1}\NormalTok{, }\DataTypeTok{max =} \DecValTok{5}\NormalTok{),}
  \CommentTok{# this time instead of a slider, we are going to define the }
  \CommentTok{# dropdown options for selecting data from the iris dataset.}
  \CommentTok{# We are going to define the input ID, Label and Choices.}
  \CommentTok{# Choices are a name you decide, and then the respective Collumn }
  \CommentTok{# name in iris. (See prind out above for more details)}
  
  \KeywordTok{sidebarLayout}\NormalTok{(}
    \KeywordTok{sidebarPanel}\NormalTok{(}
      \KeywordTok{selectInput}\NormalTok{(}\DataTypeTok{inputId =} \StringTok{"x"}\NormalTok{,}
                  \DataTypeTok{label =} \StringTok{"Choose an independent variable"}\NormalTok{,}
                  \DataTypeTok{choices =} \KeywordTok{c}\NormalTok{(}\StringTok{"Sepal Length"}\NormalTok{ =}\StringTok{ "Sepal.Length"}\NormalTok{,}
                              \StringTok{"Sepal Width"}\NormalTok{ =}\StringTok{ "Sepal.Width"}\NormalTok{,}
                              \StringTok{"Petal Length"}\NormalTok{ =}\StringTok{ "Petal.Length"}\NormalTok{,}
                              \StringTok{"Petal Width"}\NormalTok{ =}\StringTok{ "Petal.Width"}\NormalTok{))}
\NormalTok{    ),}
    \CommentTok{# repeate for y}
    \KeywordTok{sidebarPanel}\NormalTok{(}
      \KeywordTok{selectInput}\NormalTok{(}\DataTypeTok{inputId =} \StringTok{"y"}\NormalTok{,}
                  \DataTypeTok{label =} \StringTok{"Choose an dependent variable"}\NormalTok{,}
                  \DataTypeTok{choices =} \KeywordTok{c}\NormalTok{(}\StringTok{"Sepal Length"}\NormalTok{ =}\StringTok{ "Sepal.Length"}\NormalTok{,}
                              \StringTok{"Sepal Width"}\NormalTok{ =}\StringTok{ "Sepal.Width"}\NormalTok{,}
                              \StringTok{"Petal Length"}\NormalTok{ =}\StringTok{ "Petal.Length"}\NormalTok{,}
                              \StringTok{"Petal Width"}\NormalTok{ =}\StringTok{ "Petal.Width"}\NormalTok{))}
\NormalTok{    ),}
\NormalTok{  ),}
  \CommentTok{# the output will be a plot, same as the previous example }
  \KeywordTok{mainPanel}\NormalTok{(}
    \KeywordTok{plotOutput}\NormalTok{(}\DataTypeTok{outputId =} \StringTok{"Plot"}\NormalTok{)}

\NormalTok{  )}
\NormalTok{), }\DataTypeTok{server =} \ControlFlowTok{function}\NormalTok{(input, output) \{}
  \CommentTok{# now we define what part of the app is "reactive" in the server section. }
  \CommentTok{# in this case it will be the user defined variables to use for the modeling}
  \CommentTok{# and plotting.}
\NormalTok{  output}\OperatorTok{$}\NormalTok{Plot =}\StringTok{ }\KeywordTok{renderPlot}\NormalTok{(\{}
    \CommentTok{# using the user inputs, we will build the dataset used}
\NormalTok{    data =}\StringTok{ }\KeywordTok{data.frame}\NormalTok{(}\DataTypeTok{x =}\NormalTok{ iris[,input}\OperatorTok{$}\NormalTok{x],}
                      \DataTypeTok{y =}\NormalTok{ iris[,input}\OperatorTok{$}\NormalTok{y])}
    \CommentTok{# compute a linear model with  polynomial term}
\NormalTok{    model =}\StringTok{ }\KeywordTok{with}\NormalTok{(data,}\KeywordTok{lm}\NormalTok{(y }\OperatorTok{~}\StringTok{ }\KeywordTok{poly}\NormalTok{(x,input}\OperatorTok{$}\NormalTok{poly_degree)))}
    \CommentTok{# predict data out for plotting}
\NormalTok{    predict_data =}\StringTok{ }\KeywordTok{data.frame}\NormalTok{(}\DataTypeTok{x =} \KeywordTok{seq}\NormalTok{(}\KeywordTok{min}\NormalTok{(data}\OperatorTok{$}\NormalTok{x), }\KeywordTok{max}\NormalTok{(data}\OperatorTok{$}\NormalTok{x), }\DataTypeTok{length.out =} \DecValTok{1000}\NormalTok{))}
\NormalTok{    predicted.intervals =}\StringTok{ }\KeywordTok{data.frame}\NormalTok{(}\KeywordTok{predict}\NormalTok{(model, }\DataTypeTok{newdata =}\NormalTok{ predict_data,}
                                             \DataTypeTok{interval =} \StringTok{"confidence"}\NormalTok{))}
    \CommentTok{# extract some inforamtion from the lm for the plot}
\NormalTok{    summary =}\StringTok{ }\KeywordTok{summary}\NormalTok{(model)}
    \CommentTok{# build the plot itself}
    \KeywordTok{plot}\NormalTok{(data}\OperatorTok{$}\NormalTok{x, data}\OperatorTok{$}\NormalTok{y, }\DataTypeTok{xlab =}\NormalTok{ input}\OperatorTok{$}\NormalTok{x, }\DataTypeTok{ylab =}\NormalTok{ input}\OperatorTok{$}\NormalTok{y, }\DataTypeTok{col =}\NormalTok{ iris}\OperatorTok{$}\NormalTok{Species)}
    \CommentTok{# add the lm as an and confidence intervals as lines}
    \KeywordTok{lines}\NormalTok{(predict_data}\OperatorTok{$}\NormalTok{x, predicted.intervals}\OperatorTok{$}\NormalTok{fit,}\DataTypeTok{col=}\StringTok{'green'}\NormalTok{,}\DataTypeTok{lwd=}\DecValTok{3}\NormalTok{)}
    \KeywordTok{lines}\NormalTok{(predict_data}\OperatorTok{$}\NormalTok{x, predicted.intervals}\OperatorTok{$}\NormalTok{lwr,}\DataTypeTok{col=}\StringTok{'black'}\NormalTok{,}\DataTypeTok{lwd=}\DecValTok{1}\NormalTok{)}
    \KeywordTok{lines}\NormalTok{(predict_data}\OperatorTok{$}\NormalTok{x, predicted.intervals}\OperatorTok{$}\NormalTok{upr,}\DataTypeTok{col=}\StringTok{'black'}\NormalTok{,}\DataTypeTok{lwd=}\DecValTok{1}\NormalTok{)}
    \CommentTok{# add the r2 of the regression to the plot}
    \KeywordTok{mtext}\NormalTok{(}\KeywordTok{paste0}\NormalTok{(}\StringTok{"r = "}\NormalTok{, }\KeywordTok{summary}\NormalTok{(model)}\OperatorTok{$}\NormalTok{r.squared), }\DataTypeTok{side=}\DecValTok{3}\NormalTok{)}
    \CommentTok{# add a color scalling legend}
    \KeywordTok{legend}\NormalTok{(}\StringTok{"topleft"}\NormalTok{,}\DataTypeTok{legend=}\KeywordTok{levels}\NormalTok{(iris}\OperatorTok{$}\NormalTok{Species),}\DataTypeTok{col=}\DecValTok{1}\OperatorTok{:}\DecValTok{3}\NormalTok{, }\DataTypeTok{pch=}\DecValTok{1}\NormalTok{)}
\NormalTok{  \})}
\NormalTok{\})}
\end{Highlighting}
\end{Shaded}

Bingo! We have a working multivariate app that allows for some data
analysis. At this point you should start to see some of the utility of
building apps. Apps allow the user to get a much more ``data rich''
experience by allowing them to explore the data. In other words, they
can evaluate your data and analysis more on their own terms. In summary
allowing for flexibility in data visualization and analysis can promote
a much greater understanding of your research.

\hypertarget{again-with-very-minimal-comments}{%
\subsection{Again, with very minimal
comments:}\label{again-with-very-minimal-comments}}

Now try to remove the UI option to choose a polynomial degree (hint: you
will have to remove a component of the UI and modify the model section
of the server). I want you to start hacking this code so you get more
comfortable modifying the scripts.

\begin{Shaded}
\begin{Highlighting}[]
\KeywordTok{shinyApp}\NormalTok{(}\DataTypeTok{ui =} \KeywordTok{fluidPage}\NormalTok{(}
  \KeywordTok{titlePanel}\NormalTok{(}\StringTok{"Flowers!"}\NormalTok{),}
  \KeywordTok{numericInput}\NormalTok{(}\DataTypeTok{inputId =} \StringTok{"cactus"}\NormalTok{,}
                  \DataTypeTok{label =} \StringTok{"Choose Polynomial Degree"}\NormalTok{,}
                  \DataTypeTok{value =} \DecValTok{1}\NormalTok{, }\DataTypeTok{min =} \DecValTok{1}\NormalTok{, }\DataTypeTok{max =} \DecValTok{5}\NormalTok{),}
  \KeywordTok{sidebarLayout}\NormalTok{(}
    \KeywordTok{sidebarPanel}\NormalTok{(}
      \KeywordTok{selectInput}\NormalTok{(}\DataTypeTok{inputId =} \StringTok{"x"}\NormalTok{,}
                  \DataTypeTok{label =} \StringTok{"Choose an independent variable"}\NormalTok{,}
                  \DataTypeTok{choices =} \KeywordTok{c}\NormalTok{(}\StringTok{"Sepal Length"}\NormalTok{ =}\StringTok{ "Sepal.Length"}\NormalTok{,}
                              \StringTok{"Sepal Width"}\NormalTok{ =}\StringTok{ "Sepal.Width"}\NormalTok{,}
                              \StringTok{"Petal Length"}\NormalTok{ =}\StringTok{ "Petal.Length"}\NormalTok{,}
                              \StringTok{"Petal Width"}\NormalTok{ =}\StringTok{ "Petal.Width"}\NormalTok{))}
\NormalTok{    ),}
    \KeywordTok{sidebarPanel}\NormalTok{(}
      \KeywordTok{selectInput}\NormalTok{(}\DataTypeTok{inputId =} \StringTok{"y"}\NormalTok{,}
                  \DataTypeTok{label =} \StringTok{"Choose an dependent variable"}\NormalTok{,}
                  \DataTypeTok{choices =} \KeywordTok{c}\NormalTok{(}\StringTok{"Sepal Length"}\NormalTok{ =}\StringTok{ "Sepal.Length"}\NormalTok{,}
                              \StringTok{"Sepal Width"}\NormalTok{ =}\StringTok{ "Sepal.Width"}\NormalTok{,}
                              \StringTok{"Petal Length"}\NormalTok{ =}\StringTok{ "Petal.Length"}\NormalTok{,}
                              \StringTok{"Petal Width"}\NormalTok{ =}\StringTok{ "Petal.Width"}\NormalTok{))}
\NormalTok{    ),}
\NormalTok{  ),}
  \KeywordTok{mainPanel}\NormalTok{(}
    \CommentTok{#Output "Plot"}
    \KeywordTok{plotOutput}\NormalTok{(}\DataTypeTok{outputId =} \StringTok{"foobar"}\NormalTok{)}

\NormalTok{  )}
\NormalTok{), }\DataTypeTok{server =} \ControlFlowTok{function}\NormalTok{(input, output) \{}
  \CommentTok{#render a plot called "Plot"}
\NormalTok{  output}\OperatorTok{$}\NormalTok{foobar =}\StringTok{ }\KeywordTok{renderPlot}\NormalTok{(\{}
    \CommentTok{#Data}
\NormalTok{    data =}\StringTok{ }\KeywordTok{data.frame}\NormalTok{(}\DataTypeTok{x =}\NormalTok{ iris[,input}\OperatorTok{$}\NormalTok{x],}
                      \DataTypeTok{y =}\NormalTok{ iris[,input}\OperatorTok{$}\NormalTok{y])}
    \CommentTok{#Model}
\NormalTok{    model =}\StringTok{ }\KeywordTok{with}\NormalTok{(data,}\KeywordTok{lm}\NormalTok{(y }\OperatorTok{~}\StringTok{ }\KeywordTok{poly}\NormalTok{(x,input}\OperatorTok{$}\NormalTok{cactus)))}
\NormalTok{    predict_data =}\StringTok{ }\KeywordTok{data.frame}\NormalTok{(}\DataTypeTok{x =} \KeywordTok{seq}\NormalTok{(}\KeywordTok{min}\NormalTok{(data}\OperatorTok{$}\NormalTok{x), }\KeywordTok{max}\NormalTok{(data}\OperatorTok{$}\NormalTok{x), }\DataTypeTok{length.out =} \DecValTok{1000}\NormalTok{))}
\NormalTok{    predicted.intervals =}\StringTok{ }\KeywordTok{data.frame}\NormalTok{(}\KeywordTok{predict}\NormalTok{(model, }\DataTypeTok{newdata =}\NormalTok{ predict_data,}
                                             \DataTypeTok{interval =} \StringTok{"confidence"}\NormalTok{))}
    \CommentTok{#plot}
    \KeywordTok{plot}\NormalTok{(data}\OperatorTok{$}\NormalTok{x, data}\OperatorTok{$}\NormalTok{y, }\DataTypeTok{xlab =}\NormalTok{ input}\OperatorTok{$}\NormalTok{x, }\DataTypeTok{ylab =}\NormalTok{ input}\OperatorTok{$}\NormalTok{y, }\DataTypeTok{col =}\NormalTok{ iris}\OperatorTok{$}\NormalTok{Species)}
    \KeywordTok{lines}\NormalTok{(predict_data}\OperatorTok{$}\NormalTok{x, predicted.intervals}\OperatorTok{$}\NormalTok{fit,}\DataTypeTok{col=}\StringTok{'green'}\NormalTok{,}\DataTypeTok{lwd=}\DecValTok{3}\NormalTok{)}
    \KeywordTok{lines}\NormalTok{(predict_data}\OperatorTok{$}\NormalTok{x, predicted.intervals}\OperatorTok{$}\NormalTok{lwr,}\DataTypeTok{col=}\StringTok{'black'}\NormalTok{,}\DataTypeTok{lwd=}\DecValTok{1}\NormalTok{)}
    \KeywordTok{lines}\NormalTok{(predict_data}\OperatorTok{$}\NormalTok{x, predicted.intervals}\OperatorTok{$}\NormalTok{upr,}\DataTypeTok{col=}\StringTok{'black'}\NormalTok{,}\DataTypeTok{lwd=}\DecValTok{1}\NormalTok{)}
    \KeywordTok{mtext}\NormalTok{(}\KeywordTok{paste0}\NormalTok{(}\StringTok{"r = "}\NormalTok{, }\KeywordTok{summary}\NormalTok{(model)}\OperatorTok{$}\NormalTok{r.squared), }\DataTypeTok{side=}\DecValTok{3}\NormalTok{)}
    \KeywordTok{legend}\NormalTok{(}\StringTok{"topleft"}\NormalTok{,}\DataTypeTok{legend=}\KeywordTok{levels}\NormalTok{(iris}\OperatorTok{$}\NormalTok{Species),}\DataTypeTok{col=}\DecValTok{1}\OperatorTok{:}\DecValTok{3}\NormalTok{, }\DataTypeTok{pch=}\DecValTok{1}\NormalTok{)}
\NormalTok{  \})}
\NormalTok{\})}
\end{Highlighting}
\end{Shaded}

\hypertarget{remember-when-you-start-to-build-these-apps-on-your-own-and-get-stuck.-google-stack-overflow-etc-is-your-best-friend-there-are-so-many-resources-out-there-for-r-users-take-advantage-of-the-community-and-start-piecing-together-others-code-and-insight-until-you-feel-comfortable.-for-example-check-this-out-httpsshiny.rstudio.comgallery-all-of-the-code-used-to-create-these-apps-is-available-to-you.}{%
\subsubsection{\texorpdfstring{REMEMBER!! When you start to build these
apps on your own and get stuck\ldots{}. GOOGLE!!! Stack Overflow (etc)
is your BEST FRIEND!! There are so many resources out there for R users,
take advantage of the community and start piecing together others code
and insight until you feel comfortable. For example, check this out
\url{https://shiny.rstudio.com/gallery/} All of the code used to create
these apps is available to
you.}{REMEMBER!! When you start to build these apps on your own and get stuck\ldots{}. GOOGLE!!! Stack Overflow (etc) is your BEST FRIEND!! There are so many resources out there for R users, take advantage of the community and start piecing together others code and insight until you feel comfortable. For example, check this out https://shiny.rstudio.com/gallery/ All of the code used to create these apps is available to you.}}\label{remember-when-you-start-to-build-these-apps-on-your-own-and-get-stuck.-google-stack-overflow-etc-is-your-best-friend-there-are-so-many-resources-out-there-for-r-users-take-advantage-of-the-community-and-start-piecing-together-others-code-and-insight-until-you-feel-comfortable.-for-example-check-this-out-httpsshiny.rstudio.comgallery-all-of-the-code-used-to-create-these-apps-is-available-to-you.}}

\hypertarget{leaflet-introduction-and-installation}{%
\section{Leaflet: Introduction and
installation}\label{leaflet-introduction-and-installation}}

Leaflet is a very powerful library that allows for interactive mapping.
A lot of research in the natural sciences has a spatial component and
often we rely on graphical {[}G{]} user interfaces {[}UIs{]} (GUIs) to
produce visualizations of our data. Think ESRI ArcGIS. In some cases,
making quick maps for visualizing can be a pain and sometimes you want a
product that is interactive for the end user without them having to have
access to a GIS platform.

With this I present Leaflet, an open-source JavaScript library for
mobile-friendly interactive maps. But\ldots{} Now we have a R library
that brings this powerful mapping service to R users without having to
code in JavaScript (except for customization).

\hypertarget{quick-note}{%
\subsection{Quick Note!}\label{quick-note}}

In this next section I am going to be using an operator you might not be
familiar with, the ``pipe'' operator (\%\textgreater{}\%). This is a
very nifty tool that allows a user to avoid having tons of parentheses,
making code much more difficult to read. Pipes also allow you to avoid
redefining derivatives of the same data.

When you put a bunch of pipes together, complex data manipulation can be
easy to read and understand by someone not familiar with your code. The
main jist of the pipe operator is that the data set from the previous
line is passed to the next part of the ``pipe chain''.

Here is an example:

\begin{Shaded}
\begin{Highlighting}[]
\KeywordTok{library}\NormalTok{(dplyr)}

\CommentTok{# traditional way of doing some wierd data manipulation (HARD TO READ!!!)}
\KeywordTok{round}\NormalTok{(}\KeywordTok{exp}\NormalTok{(}\KeywordTok{sin}\NormalTok{(}\KeywordTok{log}\NormalTok{(iris}\OperatorTok{$}\NormalTok{Sepal.Length))),}\DecValTok{2}\NormalTok{)}

\CommentTok{# the other non preferable option. }
\NormalTok{data =}\StringTok{ }\NormalTok{iris}\OperatorTok{$}\NormalTok{Sepal.Length}
\NormalTok{log_data =}\StringTok{ }\KeywordTok{log}\NormalTok{(data)}
\NormalTok{sin_log_data =}\StringTok{ }\KeywordTok{sin}\NormalTok{(log_data)}
\NormalTok{exp_sin_log_data =}\StringTok{ }\KeywordTok{exp}\NormalTok{(sin_log_data)}
\NormalTok{round_exp_sin_log_data =}\StringTok{ }\KeywordTok{round}\NormalTok{(exp_sin_log_data,}\DecValTok{2}\NormalTok{)}
\KeywordTok{print}\NormalTok{(round_exp_sin_log_data)}

\CommentTok{# with a pipe operator (much easier to read, also allows you to avoid redefining data)}
\NormalTok{iris}\OperatorTok{$}\NormalTok{Sepal.Length }\OperatorTok
\StringTok{  }\KeywordTok{log}\NormalTok{()}\OperatorTok
\StringTok{  }\KeywordTok{sin}\NormalTok{()}\OperatorTok
\StringTok{  }\KeywordTok{exp}\NormalTok{()}\OperatorTok
\StringTok{  }\KeywordTok{round}\NormalTok{(., }\DecValTok{2}\NormalTok{)}
\end{Highlighting}
\end{Shaded}

They all yield the same result, they just do so in different ways. When
you start to have large data sets the pipe chain method saves tons of
RAM space (compared to defining each derivative product) and mixed with
packages that leverage C++ (like dplyr) make your code run much, much
faster.

Ok, back to leaflet.

\hypertarget{simple-leaflet-example}{%
\section{Simple Leaflet example}\label{simple-leaflet-example}}

Let's say you want a simple ESRI like base map.

\begin{Shaded}
\begin{Highlighting}[]
\KeywordTok{library}\NormalTok{(leaflet)}
\KeywordTok{leaflet}\NormalTok{() }\OperatorTok
\StringTok{  }\KeywordTok{addProviderTiles}\NormalTok{(providers}\OperatorTok{$}\NormalTok{Esri.NatGeoWorldMap)}
\end{Highlighting}
\end{Shaded}

DONE! How cool is that? Now we can make the map go straight to Missoula.

\begin{Shaded}
\begin{Highlighting}[]
\KeywordTok{leaflet}\NormalTok{() }\OperatorTok
\StringTok{  }\KeywordTok{addProviderTiles}\NormalTok{(providers}\OperatorTok{$}\NormalTok{Esri.NatGeoWorldMap) }\OperatorTok
\StringTok{  }\KeywordTok{setView}\NormalTok{(}\DataTypeTok{lat =} \FloatTok{46.875676}\NormalTok{, }\DataTypeTok{lng =} \FloatTok{-113.991386}\NormalTok{, }\DataTypeTok{zoom =} \DecValTok{12}\NormalTok{)}
\end{Highlighting}
\end{Shaded}

Now lets add some data you collected from the field

\begin{Shaded}
\begin{Highlighting}[]
\CommentTok{# define some data with a location and a name }
\CommentTok{# this is where you would substatute your own data from the field }
\NormalTok{data_from_the_field =}\StringTok{ }\KeywordTok{data.frame}\NormalTok{(}\DataTypeTok{names =} \KeywordTok{c}\NormalTok{(}\StringTok{"Site 1"}\NormalTok{, }\StringTok{"Site 2"}\NormalTok{, }\StringTok{"Site 3"}\NormalTok{),}
                                 \DataTypeTok{lat =} \KeywordTok{c}\NormalTok{(}\FloatTok{46.875}\NormalTok{, }\FloatTok{46.877}\NormalTok{, }\FloatTok{46.887}\NormalTok{),}
                                 \DataTypeTok{long =} \KeywordTok{c}\NormalTok{(}\OperatorTok{-}\FloatTok{113.991}\NormalTok{, }\FloatTok{-113.994}\NormalTok{, }\DecValTok{-114}\NormalTok{))}

\KeywordTok{leaflet}\NormalTok{() }\OperatorTok
\StringTok{  }\KeywordTok{addProviderTiles}\NormalTok{(providers}\OperatorTok{$}\NormalTok{Esri.NatGeoWorldMap) }\OperatorTok
\StringTok{  }\KeywordTok{setView}\NormalTok{(}\DataTypeTok{lat =} \FloatTok{46.875676}\NormalTok{, }\DataTypeTok{lng =} \FloatTok{-113.991386}\NormalTok{, }\DataTypeTok{zoom =} \DecValTok{13}\NormalTok{) }\OperatorTok
\StringTok{  }\KeywordTok{addMarkers}\NormalTok{(data_from_the_field}\OperatorTok{$}\NormalTok{long, data_from_the_field}\OperatorTok{$}\NormalTok{lat, }
             \DataTypeTok{popup =}\NormalTok{ data_from_the_field}\OperatorTok{$}\NormalTok{names)}
\end{Highlighting}
\end{Shaded}

\hypertarget{mixing-leaflet-and-shiny}{%
\section{Mixing Leaflet and Shiny}\label{mixing-leaflet-and-shiny}}

Here we are going to pull in current earthquake data from the last 30
days from the USGS (updated every minute) and plot it using Leaflet. We
will then use Shiny to allow the user to select a minimum magnitude to
crop the data and have shiny re-render the Leaflet map. Notice how
little code this is\ldots{} pretty sweet!

\begin{Shaded}
\begin{Highlighting}[]
\KeywordTok{library}\NormalTok{(dplyr)}

\CommentTok{# read in data from the USGS server}
\NormalTok{earthquakes =}\StringTok{ }\KeywordTok{read.csv}\NormalTok{(}\StringTok{"https://earthquake.usgs.gov/earthquakes/feed/v1.0/summary/2.5_month.csv"}\NormalTok{) }\OperatorTok
\StringTok{  }\KeywordTok{select}\NormalTok{(latitude, longitude, mag, depth)}

\KeywordTok{shinyApp}\NormalTok{(}\DataTypeTok{ui =} \KeywordTok{bootstrapPage}\NormalTok{(}
  \KeywordTok{titlePanel}\NormalTok{(}\StringTok{"Earthquakes!"}\NormalTok{),}
  \CommentTok{# Input Slider}
  \KeywordTok{sliderInput}\NormalTok{(}\DataTypeTok{inputId =} \StringTok{"min_mag"}\NormalTok{,}
              \DataTypeTok{label =} \StringTok{"Minimum Earthquake Magnitude"}\NormalTok{,}
              \DataTypeTok{min =} \FloatTok{2.5}\NormalTok{, }\DataTypeTok{max =} \KeywordTok{max}\NormalTok{(earthquakes}\OperatorTok{$}\NormalTok{mag),}
              \DataTypeTok{value =} \FloatTok{2.5}\NormalTok{, }\DataTypeTok{step =} \FloatTok{0.1}\NormalTok{),}
  \CommentTok{# Render the leaflet map}
  \KeywordTok{leafletOutput}\NormalTok{(}\StringTok{"mymap"}\NormalTok{, }\DataTypeTok{height =} \StringTok{"600"}\NormalTok{)}
\NormalTok{  ),}
   \DataTypeTok{server =} \ControlFlowTok{function}\NormalTok{(input, output)\{}
\NormalTok{    output}\OperatorTok{$}\NormalTok{mymap =}\StringTok{ }\KeywordTok{renderLeaflet}\NormalTok{(\{}
    \CommentTok{# Classifying the type of earthquake based on magnitude}
\NormalTok{      quakes =}\StringTok{ }\NormalTok{earthquakes[earthquakes}\OperatorTok{$}\NormalTok{mag }\OperatorTok{>}\StringTok{ }\NormalTok{input}\OperatorTok{$}\NormalTok{min_mag, ]}
      \CommentTok{# reactively define color ramp}
\NormalTok{      pal =}\StringTok{ }\KeywordTok{colorNumeric}\NormalTok{(}\DataTypeTok{palette =} \KeywordTok{c}\NormalTok{(}\StringTok{"green"}\NormalTok{, }\StringTok{"yellow"}\NormalTok{, }\StringTok{"red"}\NormalTok{), }\DataTypeTok{domain =}\NormalTok{ quakes}\OperatorTok{$}\NormalTok{mag)}
      \CommentTok{# generate map with leaflet}
      \KeywordTok{leaflet}\NormalTok{(}\DataTypeTok{data=}\NormalTok{quakes) }\OperatorTok
\StringTok{        }\KeywordTok{addProviderTiles}\NormalTok{(}\StringTok{"CartoDB.Positron"}\NormalTok{) }\OperatorTok
\StringTok{        }\KeywordTok{addCircleMarkers}\NormalTok{(}\DataTypeTok{lng=}\OperatorTok{~}\NormalTok{longitude, }\DataTypeTok{lat=}\OperatorTok{~}\NormalTok{latitude, }\DataTypeTok{weight =} \DecValTok{1}\NormalTok{, }\DataTypeTok{radius =} \DecValTok{7}\NormalTok{, }\DataTypeTok{color =} \StringTok{"black"}\NormalTok{, }\DataTypeTok{fillColor =}\OperatorTok{~}\KeywordTok{pal}\NormalTok{(mag), }\DataTypeTok{popup=}\KeywordTok{paste}\NormalTok{(}\StringTok{"Magnitude = "}\NormalTok{, quakes}\OperatorTok{$}\NormalTok{mag)) }\OperatorTok
\StringTok{        }\KeywordTok{addLegend}\NormalTok{(}\DataTypeTok{position=}\StringTok{"bottomleft"}\NormalTok{, }\DataTypeTok{pal=}\NormalTok{pal, }\DataTypeTok{values =} \OperatorTok{~}\NormalTok{mag, }\DataTypeTok{title =} \StringTok{"Magnitude"}\NormalTok{, }\DataTypeTok{opacity =} \FloatTok{0.3}\NormalTok{)}
\NormalTok{  \})}
\NormalTok{\})}
\end{Highlighting}
\end{Shaded}

\hypertarget{challenge-can-you-add-another-slider-bar-to-the-app-to-crop-data-to-a-certain-depth-as-well-as-the-current-magnitude-cropping-hint-depth-data-is-already-present-in-the-dataset.}{%
\subsection{Challenge! Can you add another slider bar to the app to crop
data to a certain depth as well as the current magnitude cropping? Hint
depth data is already present in the
dataset.}\label{challenge-can-you-add-another-slider-bar-to-the-app-to-crop-data-to-a-certain-depth-as-well-as-the-current-magnitude-cropping-hint-depth-data-is-already-present-in-the-dataset.}}

\hypertarget{shiny-leaflet-and-custom-functions}{%
\section{Shiny, Leaflet and custom
functions}\label{shiny-leaflet-and-custom-functions}}

This is where things get really cool. Coupling custom functions with
geospatial information from leaflet and using the reactive capabilities
of Shiny yield some seriously powerful tools. For example, a question I
get a lot working in the climate office is ``home much precipitation
does {[}insert place here{]} get''. I am going to share with you a tool
that can answer that question for any location in the continental U.S.
using gridMET data produced by the University of Idaho.

\begin{Shaded}
\begin{Highlighting}[]
\KeywordTok{library}\NormalTok{(ncdf4)}
\KeywordTok{library}\NormalTok{(lubridate)}
\KeywordTok{library}\NormalTok{(dplyr)}
\KeywordTok{library}\NormalTok{(zoo)}
\KeywordTok{library}\NormalTok{(ggplot2)}
\KeywordTok{library}\NormalTok{(scales)}

\NormalTok{get_precip =}\StringTok{ }\ControlFlowTok{function}\NormalTok{(lat_in, lon_in)\{}
  \CommentTok{#Define URL to net cdf }
\NormalTok{  urltotal =}\StringTok{ "http://thredds.northwestknowledge.net:8080/thredds/dodsC/agg_met_pr_1979_CurrentYear_CONUS.nc"}
  \CommentTok{# OPEN THE FILE}
\NormalTok{  nc =}\StringTok{ }\KeywordTok{nc_open}\NormalTok{(urltotal)}
  \CommentTok{# find length of time variable for extraction}
\NormalTok{  endcount =}\StringTok{ }\NormalTok{nc}\OperatorTok{$}\NormalTok{var[[}\DecValTok{1}\NormalTok{]]}\OperatorTok{$}\NormalTok{varsize[}\DecValTok{3}\NormalTok{] }
  \CommentTok{# Querry the lat lon matrix}
\NormalTok{  lon_matrix =}\StringTok{ }\NormalTok{nc}\OperatorTok{$}\NormalTok{var[[}\DecValTok{1}\NormalTok{]]}\OperatorTok{$}\NormalTok{dim[[}\DecValTok{1}\NormalTok{]]}\OperatorTok{$}\NormalTok{vals}
\NormalTok{  lat_matrix =}\StringTok{ }\NormalTok{nc}\OperatorTok{$}\NormalTok{var[[}\DecValTok{1}\NormalTok{]]}\OperatorTok{$}\NormalTok{dim[[}\DecValTok{2}\NormalTok{]]}\OperatorTok{$}\NormalTok{vals}
  \CommentTok{# find lat long that corispond}
\NormalTok{  lon=}\KeywordTok{which}\NormalTok{(}\KeywordTok{abs}\NormalTok{(lon_matrix}\OperatorTok{-}\NormalTok{lon_in)}\OperatorTok{==}\KeywordTok{min}\NormalTok{(}\KeywordTok{abs}\NormalTok{(lon_matrix}\OperatorTok{-}\NormalTok{lon_in)))  }
\NormalTok{  lat=}\KeywordTok{which}\NormalTok{(}\KeywordTok{abs}\NormalTok{(lat_matrix}\OperatorTok{-}\NormalTok{lat_in)}\OperatorTok{==}\KeywordTok{min}\NormalTok{(}\KeywordTok{abs}\NormalTok{(lat_matrix}\OperatorTok{-}\NormalTok{lat_in))) }
  \CommentTok{# define variable name}
\NormalTok{  var=}\StringTok{"precipitation_amount"}
  \CommentTok{# read data and time and extract useful time information}
\NormalTok{  data =}\StringTok{ }\KeywordTok{data.frame}\NormalTok{(}\DataTypeTok{data =} \KeywordTok{ncvar_get}\NormalTok{(nc, var, }\DataTypeTok{start=}\KeywordTok{c}\NormalTok{(lon,lat,}\DecValTok{1}\NormalTok{),}\DataTypeTok{count=}\KeywordTok{c}\NormalTok{(}\DecValTok{1}\NormalTok{,}\DecValTok{1}\NormalTok{,endcount))) }\OperatorTok
\StringTok{    }\KeywordTok{mutate}\NormalTok{(}\DataTypeTok{time =} \KeywordTok{as.Date}\NormalTok{(}\KeywordTok{ncvar_get}\NormalTok{(nc, }\StringTok{"day"}\NormalTok{, }\DataTypeTok{start=}\KeywordTok{c}\NormalTok{(}\DecValTok{1}\NormalTok{),}\DataTypeTok{count=}\KeywordTok{c}\NormalTok{(endcount)), }\DataTypeTok{origin=}\StringTok{"1900-01-01"}\NormalTok{)) }\OperatorTok
\StringTok{    }\KeywordTok{mutate}\NormalTok{(}\DataTypeTok{day =} \KeywordTok{yday}\NormalTok{(time)) }\OperatorTok
\StringTok{    }\KeywordTok{mutate}\NormalTok{(}\DataTypeTok{year =} \KeywordTok{year}\NormalTok{(time)) }\OperatorTok
\StringTok{    }\KeywordTok{mutate}\NormalTok{(}\DataTypeTok{month =} \KeywordTok{month}\NormalTok{(time, }\DataTypeTok{label =}\NormalTok{ T, }\DataTypeTok{abbr =}\NormalTok{ F))}
  \CommentTok{# close file}
  \KeywordTok{nc_close}\NormalTok{(nc)}
  
  \CommentTok{#monthly data}
\NormalTok{  monthly_data =}\StringTok{ }\NormalTok{data }\OperatorTok
\StringTok{    }\KeywordTok{group_by}\NormalTok{(month, year) }\OperatorTok
\StringTok{    }\NormalTok{dplyr}\OperatorTok{::}\KeywordTok{summarise}\NormalTok{(}\DataTypeTok{sum =} \KeywordTok{sum}\NormalTok{(data)) }\OperatorTok
\StringTok{    }\KeywordTok{mutate}\NormalTok{(}\DataTypeTok{time =} \KeywordTok{as.POSIXct}\NormalTok{(}\KeywordTok{as.Date}\NormalTok{(}\KeywordTok{as.yearmon}\NormalTok{(}\KeywordTok{paste}\NormalTok{(year, month, }\DataTypeTok{sep =} \StringTok{"-"}\NormalTok{),}
                                                \StringTok{'%Y-%b'}\NormalTok{)))) }\OperatorTok
\StringTok{    }\KeywordTok{arrange}\NormalTok{(time) }
  \CommentTok{# define ggplot function to display 3 years of data}
\NormalTok{  plot_function =}\StringTok{ }\ControlFlowTok{function}\NormalTok{(data)\{}
\NormalTok{    precip_plot =}\StringTok{ }\KeywordTok{ggplot}\NormalTok{(}\DataTypeTok{data =}\NormalTok{ data, }\KeywordTok{aes}\NormalTok{(}\DataTypeTok{x =}\NormalTok{ time, }\DataTypeTok{y =}\NormalTok{ sum))}\OperatorTok{+}
\StringTok{      }\KeywordTok{geom_bar}\NormalTok{(}\DataTypeTok{stat =} \StringTok{'identity'}\NormalTok{, }\DataTypeTok{fill =} \StringTok{"blue"}\NormalTok{)}\OperatorTok{+}
\StringTok{      }\KeywordTok{xlab}\NormalTok{(}\StringTok{""}\NormalTok{)}\OperatorTok{+}
\StringTok{      }\KeywordTok{ylab}\NormalTok{(}\StringTok{"Precipitation (mm/month)"}\NormalTok{)}\OperatorTok{+}
\StringTok{      }\KeywordTok{theme_bw}\NormalTok{(}\DataTypeTok{base_size =} \DecValTok{16}\NormalTok{)}\OperatorTok{+}
\StringTok{      }\KeywordTok{ggtitle}\NormalTok{(}\StringTok{""}\NormalTok{)}\OperatorTok{+}
\StringTok{      }\KeywordTok{theme}\NormalTok{(}\DataTypeTok{legend.position=}\StringTok{"none"}\NormalTok{,}
            \DataTypeTok{axis.text.x =} \KeywordTok{element_text}\NormalTok{(}\DataTypeTok{angle =} \DecValTok{60}\NormalTok{, }\DataTypeTok{vjust =} \FloatTok{0.5}\NormalTok{))}\OperatorTok{+}
\StringTok{      }\KeywordTok{scale_x_datetime}\NormalTok{(}\DataTypeTok{breaks =} \KeywordTok{date_breaks}\NormalTok{(}\StringTok{"3 month"}\NormalTok{), }\DataTypeTok{labels=}\KeywordTok{date_format}\NormalTok{(}\StringTok{"%b / %Y"}\NormalTok{),}
                       \DataTypeTok{limits=} \KeywordTok{as.POSIXct}\NormalTok{(}\KeywordTok{c}\NormalTok{(data}\OperatorTok{$}\NormalTok{time[}\KeywordTok{length}\NormalTok{(data}\OperatorTok{$}\NormalTok{time)}\OperatorTok{-}\DecValTok{36}\NormalTok{], }
\NormalTok{                                            data}\OperatorTok{$}\NormalTok{time[}\KeywordTok{length}\NormalTok{(data}\OperatorTok{$}\NormalTok{time)]))) }
    \KeywordTok{return}\NormalTok{(precip_plot)}
\NormalTok{  \}}
  \CommentTok{# return a list of 3 things, the plot (using the function above), daily daya and monthly data}
  \KeywordTok{return}\NormalTok{(}\KeywordTok{list}\NormalTok{(}\DataTypeTok{final_plot =} \KeywordTok{plot_function}\NormalTok{(monthly_data), }
              \DataTypeTok{daily_data =} \KeywordTok{data.frame}\NormalTok{(}\DataTypeTok{time =}\NormalTok{ data}\OperatorTok{$}\NormalTok{time, }\DataTypeTok{precipitation_mm =}\NormalTok{ data}\OperatorTok{$}\NormalTok{data),}
              \DataTypeTok{monthly_data =} \KeywordTok{data.frame}\NormalTok{(}\DataTypeTok{month =}\NormalTok{ monthly_data}\OperatorTok{$}\NormalTok{month, }
                                        \DataTypeTok{year =}\NormalTok{ monthly_data}\OperatorTok{$}\NormalTok{year,}
                                        \DataTypeTok{precipitation_mm =}\NormalTok{ monthly_data}\OperatorTok{$}\NormalTok{sum)))}
\NormalTok{\}}
\end{Highlighting}
\end{Shaded}

Now using this custom function, leaflet and shiny, lets make a map that
allows a user to click on a location and receive a precipitation plot
and download 40+ years of data (both daily and monthly).

\begin{Shaded}
\begin{Highlighting}[]
\KeywordTok{library}\NormalTok{(shiny)}
\KeywordTok{library}\NormalTok{(leaflet)}
\KeywordTok{library}\NormalTok{(leaflet.extras)}
\KeywordTok{library}\NormalTok{(ncdf4)}

\KeywordTok{shinyApp}\NormalTok{(ui <-}\StringTok{ }\KeywordTok{fluidPage}\NormalTok{(}
  \CommentTok{# build our UI defining that we want a vertical layout}
  \KeywordTok{verticalLayout}\NormalTok{(),}
  \CommentTok{# first we want to display the map}
  \KeywordTok{leafletOutput}\NormalTok{(}\StringTok{"mymap"}\NormalTok{),}
  \CommentTok{# add in a conditional message for when calculations are running. }
  \KeywordTok{conditionalPanel}\NormalTok{(}\DataTypeTok{condition=}\StringTok{"$('html').hasClass('shiny-busy')"}\NormalTok{,}
\NormalTok{                                   tags}\OperatorTok{$}\KeywordTok{div}\NormalTok{(}\StringTok{"Calculating Climatology..."}\NormalTok{,}
                                            \DataTypeTok{id=}\StringTok{"loadmessage"}\NormalTok{)),}
  \CommentTok{# display our precip plot}
  \KeywordTok{plotOutput}\NormalTok{(}\StringTok{"plot"}\NormalTok{, }\DataTypeTok{width =} \StringTok{"100%"}\NormalTok{, }\DataTypeTok{height =} \StringTok{"300px"}\NormalTok{),}
  \CommentTok{# set up download buttons for the user to download data}
  \KeywordTok{downloadButton}\NormalTok{(}\StringTok{"downloadDaily"}\NormalTok{, }\StringTok{"Download Daily Data (1979 - Present)"}\NormalTok{),}
  \KeywordTok{downloadButton}\NormalTok{(}\StringTok{"downloadMonthly"}\NormalTok{, }\StringTok{"Download Monthly Data (1979 - Present)"}\NormalTok{)}
\NormalTok{  ),}
  \CommentTok{# now on to the server}
\NormalTok{  server <-}\StringTok{ }\ControlFlowTok{function}\NormalTok{(input, output, session) \{}
  \CommentTok{# this is our map that we will display}
\NormalTok{  output}\OperatorTok{$}\NormalTok{mymap <-}\StringTok{ }\KeywordTok{renderLeaflet}\NormalTok{(\{}
    \KeywordTok{leaflet}\NormalTok{() }\OperatorTok
\StringTok{    }\CommentTok{# this is the base map  }
\StringTok{    }\NormalTok{leaflet}\OperatorTok{::}\KeywordTok{addProviderTiles}\NormalTok{(}\StringTok{"Stamen.Toner"}\NormalTok{) }\OperatorTok
\StringTok{    }\CommentTok{# terrain tiles}
\StringTok{    }\NormalTok{leaflet}\OperatorTok{::}\KeywordTok{addTiles}\NormalTok{(}\StringTok{"https://maps.tilehosting.com/data/hillshades/\{z\}/\{x\}/\{y\}.png?key=KZO7rAv96Alr8UVUrd4a"}\NormalTok{) }\OperatorTok
\StringTok{    }\CommentTok{# lines and labels}
\StringTok{    }\NormalTok{leaflet}\OperatorTok{::}\KeywordTok{addProviderTiles}\NormalTok{(}\StringTok{"Stamen.TonerLines"}\NormalTok{) }\OperatorTok
\StringTok{    }\NormalTok{leaflet}\OperatorTok{::}\KeywordTok{addProviderTiles}\NormalTok{(}\StringTok{"Stamen.TonerLabels"}\NormalTok{) }\OperatorTok
\StringTok{    }\CommentTok{# set default viewing location and zoom}
\StringTok{    }\NormalTok{leaflet}\OperatorTok{::}\KeywordTok{setView}\NormalTok{(}\DataTypeTok{lng =} \FloatTok{-97.307564}\NormalTok{, }\DataTypeTok{lat =} \FloatTok{40.368971}\NormalTok{, }\DataTypeTok{zoom =} \DecValTok{4}\NormalTok{) }\OperatorTok
\StringTok{    }\CommentTok{# modify some parameters (what tools are displayed with the map)}
\StringTok{      }\NormalTok{leaflet.extras}\OperatorTok{::}\KeywordTok{addDrawToolbar}\NormalTok{(}\DataTypeTok{markerOptions =} \KeywordTok{drawMarkerOptions}\NormalTok{(),}
                   \DataTypeTok{polylineOptions =} \OtherTok{FALSE}\NormalTok{, }\DataTypeTok{polygonOptions =} \OtherTok{FALSE}\NormalTok{,}
                   \DataTypeTok{circleOptions =} \OtherTok{FALSE}\NormalTok{, }\DataTypeTok{rectangleOptions =} \OtherTok{FALSE}\NormalTok{,}
                   \DataTypeTok{circleMarkerOptions =} \OtherTok{FALSE}\NormalTok{, }\DataTypeTok{editOptions =} \OtherTok{FALSE}\NormalTok{,}
                   \DataTypeTok{singleFeature =} \OtherTok{FALSE}\NormalTok{, }\DataTypeTok{targetGroup=}\StringTok{'draw'}\NormalTok{)}
\NormalTok{  \})}
  \CommentTok{# Now for our reactive portion which is when the user drops a pin on the map}
  \KeywordTok{observeEvent}\NormalTok{(input}\OperatorTok{$}\NormalTok{mymap_draw_new_feature,\{}
      \CommentTok{# create a variable "feature" that will be overwritten when pin drops}
\NormalTok{      feature =}\StringTok{ }\NormalTok{input}\OperatorTok{$}\NormalTok{mymap_draw_new_feature}
      \CommentTok{# call our precip function and store the outputs as a variable}
\NormalTok{      function_out =}\StringTok{ }\KeywordTok{get_precip}\NormalTok{(feature}\OperatorTok{$}\NormalTok{geometry}\OperatorTok{$}\NormalTok{coordinates[[}\DecValTok{2}\NormalTok{]],}
\NormalTok{                                feature}\OperatorTok{$}\NormalTok{geometry}\OperatorTok{$}\NormalTok{coordinates[[}\DecValTok{1}\NormalTok{]])}
      \CommentTok{# render the plot from our function output}
\NormalTok{      output}\OperatorTok{$}\NormalTok{plot <-}\StringTok{ }\KeywordTok{renderPlot}\NormalTok{(\{}
\NormalTok{        function_out[[}\DecValTok{1}\NormalTok{]]}
\NormalTok{      \})}
      \CommentTok{# render the daily data output from our function to a csv for download }
      \CommentTok{# with a reactive name (lat long)}
\NormalTok{      output}\OperatorTok{$}\NormalTok{downloadDaily <-}\StringTok{ }\KeywordTok{downloadHandler}\NormalTok{(}
        \DataTypeTok{filename =} \ControlFlowTok{function}\NormalTok{() \{}
          \KeywordTok{paste}\NormalTok{(}\StringTok{"daily_precip_"}\NormalTok{,}\KeywordTok{round}\NormalTok{(feature}\OperatorTok{$}\NormalTok{geometry}\OperatorTok{$}\NormalTok{coordinates[[}\DecValTok{2}\NormalTok{]],}\DecValTok{4}\NormalTok{),}\StringTok{"_"}\NormalTok{,}
                \KeywordTok{round}\NormalTok{(feature}\OperatorTok{$}\NormalTok{geometry}\OperatorTok{$}\NormalTok{coordinates[[}\DecValTok{1}\NormalTok{]],}\DecValTok{4}\NormalTok{),}\StringTok{".csv"}\NormalTok{, }\DataTypeTok{sep =} \StringTok{""}\NormalTok{)}
\NormalTok{        \},}
        \DataTypeTok{content =} \ControlFlowTok{function}\NormalTok{(file) \{}
          \KeywordTok{write.csv}\NormalTok{(function_out}\OperatorTok{$}\NormalTok{daily_data, file, }\DataTypeTok{row.names =} \OtherTok{FALSE}\NormalTok{)}
\NormalTok{        \}}
\NormalTok{      )}
      \CommentTok{# render the monthly data output again with a reactive name}
\NormalTok{      output}\OperatorTok{$}\NormalTok{downloadMonthly <-}\StringTok{ }\KeywordTok{downloadHandler}\NormalTok{(}
        \DataTypeTok{filename =} \ControlFlowTok{function}\NormalTok{() \{}
          \KeywordTok{paste}\NormalTok{(}\StringTok{"monthly_sum_precip_"}\NormalTok{,}\KeywordTok{round}\NormalTok{(feature}\OperatorTok{$}\NormalTok{geometry}\OperatorTok{$}\NormalTok{coordinates[[}\DecValTok{2}\NormalTok{]],}\DecValTok{4}\NormalTok{),}\StringTok{"_"}\NormalTok{,}
                \KeywordTok{round}\NormalTok{(feature}\OperatorTok{$}\NormalTok{geometry}\OperatorTok{$}\NormalTok{coordinates[[}\DecValTok{1}\NormalTok{]],}\DecValTok{4}\NormalTok{),}\StringTok{".csv"}\NormalTok{, }\DataTypeTok{sep =} \StringTok{""}\NormalTok{)}
\NormalTok{        \},}
        \DataTypeTok{content =} \ControlFlowTok{function}\NormalTok{(file) \{}
          \KeywordTok{write.csv}\NormalTok{(function_out}\OperatorTok{$}\NormalTok{monthly_data, file, }\DataTypeTok{row.names =} \OtherTok{FALSE}\NormalTok{)}
\NormalTok{        \}}
\NormalTok{      )}
\NormalTok{    \})}
\NormalTok{\})}
\end{Highlighting}
\end{Shaded}

Bingo! Hopefully now you can see how powerful these packages are,
especially when combined. Another nice thing about both of these
packages is that they are ``web ready'', meaning you can host shiny apps
on the web with your own web server when combined with ``shiny-server''
or use an online system like \url{https://www.shinyapps.io}. Futher,
leaflet maps (when not combined with shiny) can be saved as HTML
documents that are ready to host on a traditional web server like Apache
or NGINX. You can also send HTML documents containing your interactive
map within an email. If you have a shiny component to your leaflet map
you will still need a shiny compatible web server.

If you have any questions about creating your own apps or maps, please
feel free to contact me. My email address is at the top of this
document.


\end{document}
